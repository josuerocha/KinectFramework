\documentclass[12pt]{article}
\usepackage{lipsum}
\usepackage{sbc-template}

\usepackage{graphicx,url}

\usepackage[brazil]{babel}   
%\usepackage[latin1]{inputenc}  
\usepackage[utf8]{inputenc}  

     
\sloppy

\title{Estudo da Point Cloud Library aplicado ao reconhecimento de objetos em nuvem de pontos 3D}

\author{Josué R. Lima\inst{1}, Odilon Corrêa\inst{1}}


\address{Centro Federal de Educação Tecnológica de Minas Gerais
  (CEFET-MG)\\
  R. Dezenove de Novembro, 121 - Centro, Timóteo - MG, 35180-008 -- Timóteo -- MG -- Brasil
  \email{josuerocha@me.com, odilon.correa@gmail.com}
}

\begin{document} 

\maketitle

\begin{abstract}
Despite having great industrial and commercial applicability, enabling a computer to recognize objects in the real world continues being one of the greatest challenges in computing. In this work a study on the Point Cloud Library (PCL) was performed, aiming at applying methods and algorithms to object recognition in 3-dimensional point clouds obtained in real time from an RGB-D camera. PCL is a  cross-platform library for 2-dimensional and 3-dimensional image processing and also n-dimensional point clouds. It comprises several algorithms for image filtering, surface reconstruction and segmentation. The study produced an analysis of the library and its functions, detailing their use, application and source-code. Through use case studies, the functions that were studied were combined in order to demonstrate both the resources and efficiency of PCL library.
Keywords: Point Cloud Library, point cloud, image processing
\end{abstract}
     
\begin{resumo} 

Apesar de ter grande aplicação industrial e comercial, fazer com que o computador seja capaz de
reconhecer objetos no mundo real continua sendo um dos grandes desafios da computação.
Neste trabalho foi realizado um estudo sobre a biblioteca Point Cloud Library (PCL), visando a
aplicação de métodos e algoritmos no reconhecimento de objetos em nuvens de pontos 3D
capturadas em tempo real a partir de uma câmera RGB-D. A PCL é uma biblioteca
multiplataforma para processamento de imagem 2D/3D e de nuvem de pontos (Point Cloud) de n
dimensões. A biblioteca possui diversos algoritmos para filtragem de imagens, reconstrução de
superfícies e segmentação. Durante o estudo foi produzida uma análise da biblioteca e suas funções,
detalhando seu uso, aplicações e códigos-fonte. Por meio de estudos de caso,
as funções estudadas foram combinadas de modo a evidenciar os recursos e eficiência da PCL.
Palavras-chave: Point Cloud Library, Nuvem de pontos, Processamento de Imagens
  		
\end{resumo}


\section{Introdução}
Por muitos anos, os sensores utilizados na robótica foram câmeras de duas dimensões (2D), que somente capturam dados RGB de uma cena. No entanto, através da análise de dados necessários para tomada de decisões relacionadas à movimento e percepção percebeu-se a necessidade de algo além.

Com a maior disponibilidade de \textbf{sensores 3D e RGB-D} obteve-se a enorme vantagem que é a facilidade de medição de distâncias. Deste modo, devido à presença de mais uma dimensão se faz necessário a pesquisa e desenvolvimento de métodos funcionais e viáveis para o tratamento dos dados obtidos.

\section{Background} \label{sec:background}


\subsection{Sensores de profundidade}

Como mencionado anteriormente, os \textbf{sensores de profundidade} fornecem informações precisas acerca das distâncias dos pontos tridimensionais presentes em uma cena. Diversos sensores estão presentes no mercado tornando assim necessário avaliar os objetivos da aplicação para fazer uma boa escolha quanto à qual sensor utilizar. Neste projeto foi utilizado o sensor \textit{Kinect} da \textit{Microsoft} por possuir qualidade e custo ideais para esta aplicação.

\begin{itemize}
  
\item \textbf{Câmeras stereo}

As câmeras stereo trabalham com a montagem de duas câmeras iguais à uma distância conhecida entre sí, e através de processamento de imagem aplicado às imagens obtidas obtem-se a distância na cena. As câmeras stereo possuem preço acessível, no entanto falham quanto a precisão.

IMAGEM

\item \textbf{Câmeras LIDAR}

As câmeras LIDAR tabalham com o escaneamento de cada ponto de uma cena através da medição do tempo de reflexão de um feixe de \textit{laser}. São colocadas sobre uma montagem móvel que direciona a câmera para cada ponto a ser escaneado. Este tipo de câmera pode tornar a aplicação bastante lenta devido à dependência de componentes mecânicos em seu funcionamento, apesar de possuir alta precisão e resolução.

IMAGEM

\item \textbf{Câmeras Time-of-Flight}

As câmeras \textit{Time-of-flight} (ToF) se diferenciam das LIDAR por obter todos os valores de profundidade de uma cena de uma só vez utilizando um pulso de luz emitido uma vez por quadro. Deste modo, o funcionamento das câmeras ToF é muito mais rápido podendo obter um índice de quadros por segundo (FPS) de 100 Hz. No entanto, possuem resolução baixa.

IMAGEM

\item \textbf{Câmeras por luz estruturada}

As câmeras por luz estruturada funcionam através da emissão de feixes de luz infravermelha de maneira organizada. Através da analize da distorção observada da luz obtem-se informação acerca da profundidade, possibilitando assim a reconstrução de superfícies. Estas câmeras possuem resolução e imagem similares às cameras RGB convencionais, frequentemente 640x480 pixels à 100 FPS.

O sensor utilizado neste projeto, o \textit{Microsoft Kinect} se enquadra neste tipo de câmera e trabalha com resolução de 640x480 pixels, 30 FPS e alcance máximo variando de 3 à 6 metros.

IMAGEM
\end{itemize}

\subsection{Nuvens de pontos}

Os dados oriundos dos sensores 3D mencionados previamente são retornados como nuvens de pontos, as quais são um grupo organizado de pontos cada qual contendo um sistema de coordenadas XYZ e também informação acerca da cor de tal ponto, que é armazenado em componentes RGB.

\subsection{Point Cloud Library}

A \textit{Point Cloud Library} (PCL) é uma biblioteca mutiplataforma e de código-fonte aberto utilizada em processamento de imagens tridimensionais e bidimensionais. O projeto foi iniciado em 2010 pelo grupo Willow Garage que também detem os projetos \textit{Robot Operating System} (ROS) e \textit{OpenCV}. O projeto também é mantido pela organização sem fins lucrativos chamada \textit{Open Perception}.

A biblioteca possui diversos submódulos bem dividos para o processamento de imagens, como visualização, filtragem, segmentação, registro, pesquisa e estimação, dentre outras.

\section{Desenvolvimento} \label{sec:desenvolvimento}
\subsection{Instalação da Point Cloud Library}
Devido à grande dificuldade de instalação da biblioteca, tanto devido ao difícil acesso à informação e ao grande número de dependências que a aplicação precisa para funcionar, foi criado um script automatizado para a instalação dos módulos necessários em ambientes Linux e reunidos os instaladores necessários para utilizar a PCL em ambientes Windows. 

Neste script, algumas dependências foram obtidas de repositórios certificados e outras foram compiladas a partir do código fonte. O script contempla as seguintes dependências:
\begin{itemize}
\item \textbf{Virtualization Toolkit (VTK):} ACRESCENTAR FUNÇÂO
\item \textbf{Eigen 3:} ACRESCENTAR FUNÇÂO
\item \textbf{Freeglut:} ACRESCENTAR FUNÇÂO
\item \textbf{OpenGL:} renderização de imagens para visualização.
\item \textbf{Boost Library:} gerenciamento das threads utilizadas pela PCL.
\item \textbf{Java Development Kit (JDK):} ACRESCENTAR FUNÇÂO
\item \textbf{Java Runtime Environment (JRE):} ACRESCENTAR FUNÇÂO
\item \textbf{XMU Library:} ACRESCENTAR FUNÇÂO
\item \textbf{USB Library:} acesso à dispositivos USB.
\item \textbf{Mono:} implementação multiplataforma do .NET.
\item \textbf{Doxygen:} ferramenta para geração de documentação à partir de código em C++.
\item \textbf{Fast Library for Aproximate Nearest Neighbors (FLANN):} cálculos de vizinhança aproximada em espaços de grandes dimensões.
\item \textbf{OpenNI ou OpenNI2:} ACRESCENTAR FUNÇÂO
\item \textbf{Sensor Kinect:} Driver para comunicação do hardware Kinect com o OpenNI.
\end{itemize}

\section{Trabalhos relacionados} \label{sec:trabalhosrelacionados}

\section{Resultados} \label{sec:resultados}

\section{Conclusão}

\section{Referências} 

\bibliographystyle{sbc}
\bibliography{sbc-template}


\end{document}
